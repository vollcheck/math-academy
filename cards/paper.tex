\documentclass{article}
\usepackage{amsmath,amssymb,amsthm,textcomp}

\title{On Math}
\author{Jacek Walczak}
\date{February - March 2025}

\begin{document}

\maketitle

\section{Default symbols}

A - area

\section{Trigonometry}

\begin{equation}
  \begin{gathered}
    \sin\theta = \frac{\text{opposite}}{\text{hypotenuse}} \\
    \cos\theta = \frac{\text{adjacent}}{\text{hypotenuse}} \\
    \tan\theta = \frac{\text{opposite}}{\text{adjacent}} \\
    \csc\theta = \frac{1}{\sin{\theta}} =
    \frac{\text{hypotenuse}}{\text{opposite}} \\
    \sec\theta = \frac{1}{\cos{\theta}} =
    \frac{\text{hypotenuse}}{\text{adjacent}} \\
    \cot\theta = \frac{1}{\tan{\theta}} =
    \frac{\text{adjacent}}{\text{opposite}} \\
  \end{gathered}
\end{equation}

\subsection{Sine values}
\begin{equation}
  \begin{gathered}
    \sin(0) = 0 \\
    \sin(30) = \frac{1}{2} \\
    \sin(45) = \frac{\sqrt{2}}{2} \\
    \sin(60) = \frac{\sqrt{3}}{2} \\
    \sin(90) = 1
  \end{gathered}
\end{equation}

\subsection{Cosine values}
\begin{equation}
  \begin{gathered}
    \cos(0) = 1 \\
    \cos(30) = \frac{\sqrt{3}}{2} \\
    \cos(45) = \frac{\sqrt{2}}{2} \\
    \cos(60) = \frac{1}{2} \\
    \cos(90) = 0
  \end{gathered}
\end{equation}

% Tangent values

\subsection{Tangent values}
\begin{equation}
  \begin{gathered}
    \tan(0) = 0 \\
    \tan(30) = \frac{1}{\sqrt{3}} = \frac{\sqrt{3}}{3} \\
    \tan(45) = 1 \\
    \tan(60) = \sqrt{3} \\
    \tan(90) = \text{undefined}
  \end{gathered}
\end{equation}

\subsection{Formula of a general triangle}

\begin{equation}
  A = \frac{1}{2}ab \sin{C}
\end{equation}
where \textit{C} is the measure of the angle between sides \textit{a} and \textit{b} (also called \textit{included angle})

\section{Complex numbers}
\subsection{Argument of a complex number}
There is natural angle $\theta$ that \textit{z} makes with real x-axis. This angle is called \textit{the argument of z}. The numeric value of $arg(z)$ is given in radians and also
\begin{equation}
0 \leq arg(z) \leq 2\pi
\end{equation}

Example:
\begin{equation}
  \begin{gathered}
  z = 3 + 4i \\
  \tan \theta = \frac{4}{3} \\
  \theta = \arctan\frac{4}{3} \approx 0.93
  \end{gathered}
\end{equation}

\section{Geometry}
\subsection{Rigid motions}
Or sometimes - any transformations that does not change the distances
and angles between points in a figure. They are ANGLE-PRESERVING and
DISTANCE-PRESERVING.

Rigid motions types:
- translations
- rotations
- reflections

Not a rigid motions:
- dilations
- stretches


\section{Functions}
\subsection{Quadratic formula}
\begin{equation}
  x = \frac{-b \pm \sqrt{b^2 - 4ac}}{2a}
\end{equation}

\subsection{Factor theorem}
For polynomial, if we know root, we can determine a factor of the
polynomial. If $p(x)$ is a polynomial and $x=r$ is a root of $p(x)$ then
\begin{equation}
  \begin{gathered}
  p(r) = 0 \\
  so (x-r) is a factor of p(x) \\
  generally: (ax-b) is a factor of a polynomial p(x) then \\
  x = \frac{b}{a} is a root of p(x)
\end{gathered}
\end{equation}

\subsection{Inverse functions}
Inverse functions are denoted by $f^{-1}(x)$.

For $f(x)$ that maps $x=2$ to $y=0$ an inverse function $f^{-1}(x)$ instantiation would be
$f^{-1}(0)=0$.

Generally it is a simple composition:
\begin{equation}
  \begin{gathered}
    f^{-1}(f(x)) = x \\
    so \\
    (f^{-1} \circ f)(x) = x
  \end{gathered}
\end{equation}

Calculate inverse function
1. swap $x$ and $y$
2. solve for $y$

Example:
\begin{equation}
  \begin{gathered}
    f(x) = 3x-9 \\
    y = 3x-9 \\
    x = 3y-9 \\
    x-3y=-9 \\
    x+9=3y \\
    \frac{x+9}{3}=y \\
    f^{-1}(x)=\frac{x+9}{3}
  \end{gathered}
\end{equation}

Domain and range of inverse function:
Domain of a function $f(x)$ is the range of its inverse function $f^{-1}(x)$.
Range of a function $f(x)$ is the domain of its inverse function $f^{-1}(x)$.

\subsection{Multiplicities of the roots of a polynomial}
\begin{equation}
  p(x) = (x-5)^2(x+8)
\end{equation}
for that polynomial roots are $x=5$ and $x=-8$ but factor $(x-5)^2$
appears TWICE so it is a MULTIPLE ROOT
$(x+8)$ on the other hand is a SIMPLE ROOT.



\subsection{Rational functions}
Horizontal asymptotes
1. Identify the 'dominant term' which is a leading term of the polynomial though we
usually ugnore the coefficients.

2. Divide every term in the numerator and denominator of $f(x)$ by the dominant term.

3. Evaluate the remaining expression as $x \to \infty$.


Roots


\subsection{Radical function}
Range of a radical function $f(x) = \sqrt[n]{x}$ depends on index \textit{n} whether it's even or odd.

For odd index like in function $f(x) = -2\sqrt[3]{3x-2}+2$ range is $(-\infty, \infty)$.
Transformations do not change its range.

For even index $n$ range is $x \geqslant 0$ but the transformations can change the range.
Horizontal shifts and stretches doesn't change the the range.

For example for function $g(x) = -3\sqrt[4]{2x-1}-2$:
\begin{equation}
  \begin{gathered}
  \sqrt[4]{2x-1} \geqslant 0 \\
  -3\sqrt[4]{2x-1} \leqslant 0 \\
  -3\sqrt[4]{2x-1} \leqslant -2 \\
  g(x) \leqslant -2
  \end{gathered}
\end{equation}

\subsection{Exponential functions}
\subsection{Logarithmic functions}
\section{Statistics}
\section{Logarithms}
- product rule
\begin{equation}
\end{equation}

- quotient rule
\begin{equation}
\end{equation}

- power rule
\begin{equation}
\end{equation}

- change of base
\begin{equation}
\end{equation}

- equating arguments
\begin{equation}
\end{equation}

\section{Finite sequences}
Coefficient rule:
\begin{equation}
  \sum_{i=1}^{n} ca_i = c * \sum_{i=1}^{n} a_i
\end{equation}

Sum rule:
\begin{equation}
  \sum_{i=1}^{n} (a_i+b_i) = \sum_{i=1}^{n} a_i + \sum_{i=1}^{n} b_i
\end{equation}

\section{Polynomial identities}
\subsection{Binomial formulas}


\subsection{Cubic formulas}

\section{Derivatives}
Average rate of change of the function $f(x)$ on the interval $[a, a+h]$

\begin{equation}
  \frac{\Delta y}{\Delta x} = \frac{f(a+h) - f(a)}{h}
\end{equation}

Instanteous rate of change of the function at the point $x=a$ is the limit of the average
rate of change as $h \rightarrow 0$. It's denoted by $f'(a)$ and given by:

\begin{equation}
  f'(a) = \lim_{h \rightarrow 0}\frac{f(a+h) - f(a)}{h}
\end{equation}

and its value is equal to the slope of the tangent line to the curve $y = f(x)$ at $x=a$.
Also it can be defined as:
\begin{equation}
  f'(a) = \lim_{x \rightarrow a}\frac{f(x) - f(a)}{x-a}
\end{equation}

\section{Integrals}
Also called \textit{antiderivatives}. To represent integrals


Integral of a sum of functions
\begin{equation}
  \int f(x) \pm g(x) \, dx = \int f(x) \, dx \pm \int g(x) \, dx
\end{equation}

Integral of a reciprocal function
\begin{equation}
  \int \frac{1}{x} \, dx = \int \ln abs(x) + C
\end{equation}
and also:
\begin{equation}
  \int \frac{1}{x} \, dx = \int \ln|x| + C = \ln abs(x) + \ln K = \ln (K (abs(x))
\end{equation}

Integral of an exponential function
\begin{equation}
  \int e^x \, dx = e^x + C
\end{equation}
with general base we know that
\begin{equation}
  \frac{d}{dx} (a^x) = a^x \ln a
\end{equation}
so reversing:
\begin{equation}
  \int a^x \, dx = \frac{a^x}{\ln a} + C
\end{equation}

Marked integral for future
\begin{equation}
  \int_{a}^{b} f(x) \, dx
\end{equation}



\end{document}
