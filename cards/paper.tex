\documentclass{article}
\usepackage{amsmath,amssymb,amsthm,textcomp}

\title{On Math}
\author{Jacek Walczak}
\date{February - March 2025}

\makeatletter
\newenvironment{leftequation}{%
  \begin{equation}%
    \hspace*{-\displaywidth}%
  }{%
  \end{equation}%
}
\makeatother


\begin{document}

\maketitle

\section{Introduction}
Math content here...

inline math: $x^2 + y^2 = z^2$
display math: \begin{equation}x^2 + y^2 = z^2 \end{equation}

\section{Default symbols}

A - area

\section{Trigonometry}

\begin{equation}
  \begin{gathered}
    \sin\alpha = \frac{\text{opposite}}{\text{hypotenuse}} \\
    \cos\alpha = \frac{\text{adjacent}}{\text{hypotenuse}} \\
    \tan\alpha = \frac{\text{opposite}}{\text{adjacent}} \\
    \csc\alpha = \frac{1}{\sin{\alpha}} =
    \frac{\text{hypotenuse}}{\text{opposite}} \\
    \sec\alpha = \frac{1}{\cos{\alpha}} =
    \frac{\text{hypotenuse}}{\text{adjacent}} \\
    \cot\alpha = \frac{1}{\tan{\alpha}} =
    \frac{\text{adjacent}}{\text{opposite}} \\
  \end{gathered}
\end{equation}

\subsection{Sine values}
\begin{equation}
  \begin{gathered}
    \sin(0°) = 0 \\
    \sin(30°) = \frac{1}{2} \\
    \sin(45°) = \frac{\sqrt{2}}{2} \\
    \sin(60°) = \frac{\sqrt{3}}{2} \\
    \sin(90°) = 1
  \end{gathered}
\end{equation}

\subsection{Cosine values}
\begin{equation}
  \begin{gathered}
    \cos(0°) = 1 \\
    \cos(30°) = \frac{\sqrt{3}}{2} \\
    \cos(45°) = \frac{\sqrt{2}}{2} \\
    \cos(60°) = \frac{1}{2} \\
    \cos(90°) = 0
  \end{gathered}
\end{equation}

% Tangent values

\subsection{Tangent values}
\begin{equation}
  \begin{gathered}
    \tan(0°) = 0 \\
    \tan(30°) = \frac{1}{\sqrt{3}} = \frac{\sqrt{3}}{3} \\
    \tan(45°) = 1 \\
    \tan(60°) = \sqrt{3} \\
    \tan(90°) = \text{undefined}
  \end{gathered}
\end{equation}

\subsection{Formula of a general triangle}

\begin{equation}
  A = \frac{1}{2}ab \sin{C}
\end{equation}
where 'C' (TODO: italize all the 'things') is the measure of the angle between sides 'a' and 'b' (also called 'included angle)

\section{Complex numbers}
\section{Functions}
\subsection{Quadratic formula}
\begin{equation}
  x = \frac{-b \pm \sqrt{b^2 - 4ac}}{2a}
\end{equation}
\subsection{Rational functions}
1. asymptotes
2. roots
\subsection{Exponential functions}
\subsection{Logarithmic functions}
\section{Statistics}
\section{Logarithms}
- product rule
\begin{equation}
\end{equation}

- quotient rule
\begin{equation}
\end{equation}

- power rule
\begin{equation}
\end{equation}

- change of base
\begin{equation}
\end{equation}

- equating arguments
\begin{equation}
\end{equation}

\section{Polynomial identities}
\subsection{Binomial formulas}


\subsection{Cubic formulas}


\end{document}
