\documentclass{article}
\usepackage{amsmath,amssymb,amsthm,textcomp}

\title{On Math}
\author{Jacek Walczak}
\date{February - March 2025}

\begin{document}

\maketitle

\section{Default symbols}

A - area

\section{Trigonometry}

\begin{equation}
  \begin{gathered}
    \sin\theta = \frac{\text{opposite}}{\text{hypotenuse}} \\
    \cos\theta = \frac{\text{adjacent}}{\text{hypotenuse}} \\
    \tan\theta = \frac{\text{opposite}}{\text{adjacent}} \\
    \csc\theta = \frac{1}{\sin{\theta}} =
    \frac{\text{hypotenuse}}{\text{opposite}} \\
    \sec\theta = \frac{1}{\cos{\theta}} =
    \frac{\text{hypotenuse}}{\text{adjacent}} \\
    \cot\theta = \frac{1}{\tan{\theta}} =
    \frac{\text{adjacent}}{\text{opposite}} \\
  \end{gathered}
\end{equation}

\subsection{Sine values}
\begin{equation}
  \begin{gathered}
    \sin(0) = 0 \\
    \sin(30) = \frac{1}{2} \\
    \sin(45) = \frac{\sqrt{2}}{2} \\
    \sin(60) = \frac{\sqrt{3}}{2} \\
    \sin(90) = 1
  \end{gathered}
\end{equation}

\subsection{Cosine values}
\begin{equation}
  \begin{gathered}
    \cos(0) = 1 \\
    \cos(30) = \frac{\sqrt{3}}{2} \\
    \cos(45) = \frac{\sqrt{2}}{2} \\
    \cos(60) = \frac{1}{2} \\
    \cos(90) = 0
  \end{gathered}
\end{equation}

% Tangent values

\subsection{Tangent values}
\begin{equation}
  \begin{gathered}
    \tan(0) = 0 \\
    \tan(30) = \frac{1}{\sqrt{3}} = \frac{\sqrt{3}}{3} \\
    \tan(45) = 1 \\
    \tan(60) = \sqrt{3} \\
    \tan(90) = \text{undefined}
  \end{gathered}
\end{equation}

\subsection{Formula of a general triangle}

\begin{equation}
  A = \frac{1}{2}ab \sin{C}
\end{equation}
where \textit{C} is the measure of the angle between sides \textit{a} and \textit{b} (also called \textit{included angle})

\section{Complex numbers}
\subsection{Argument of a complex number}
There is natural angle $\theta$ that \textit{z} makes with real x-axis. This angle is called \textit{the argument of z}. The numeric value of $arg(z)$ is given in radians and also
\begin{equation}
0 \leq arg(z) \leq 2\pi
\end{equation}

Example:
\begin{equation}
  \begin{gathered}
  z = 3 + 4i \\
  \tan \theta = \frac{4}{3} \\
  \theta = \arctan\frac{4}{3} \approx 0.93
  \end{gathered}
\end{equation}

\section{Functions}
\subsection{Quadratic formula}
\begin{equation}
  x = \frac{-b \pm \sqrt{b^2 - 4ac}}{2a}
\end{equation}

\subsection{Rational functions}
Horizontal asymptotes
1. Identify the 'dominant term' which is a leading term of the polynomial though we
usually ugnore the coefficients.

2. Divide every term in the numerator and denominator of $f(x)$ by the dominant term.

3. Evaluate the remaining expression as $x \to \infty$.


Roots


\subsection{Radical function}
Range of a radical function $f(x) = \sqrt[n]{x}$ depends on index \textit{n} whether it's even or odd.

For odd index like in function $f(x) = -2\sqrt[3]{3x-2}+2$ range is $(-\infty, \infty)$.
Transformations do not change its range.

For even index $n$ range is $x \geqslant 0$ but the transformations can change the range.
Horizontal shifts and stretches doesn't change the the range.

For example for function $g(x) = -3\sqrt[4]{2x-1}-2$:
\begin{equation}
  \begin{gathered}
  \sqrt[4]{2x-1} \geqslant 0 \\
  -3\sqrt[4]{2x-1} \leqslant 0 \\
  -3\sqrt[4]{2x-1} \leqslant -2 \\
  g(x) \leqslant -2
  \end{gathered}
\end{equation}

\subsection{Exponential functions}
\subsection{Logarithmic functions}
\section{Statistics}
\section{Logarithms}
- product rule
\begin{equation}
\end{equation}

- quotient rule
\begin{equation}
\end{equation}

- power rule
\begin{equation}
\end{equation}

- change of base
\begin{equation}
\end{equation}

- equating arguments
\begin{equation}
\end{equation}

\section{Finite sequences}
Coefficient rule:
\begin{equation}
  \sum_{i=1}^{n} ca_i = c * \sum_{i=1}^{n} a_i
\end{equation}

Sum rule:
\begin{equation}
  \sum_{i=1}^{n} (a_i+b_i) = \sum_{i=1}^{n} a_i + \sum_{i=1}^{n} b_i
\end{equation}

\section{Polynomial identities}
\subsection{Binomial formulas}


\subsection{Cubic formulas}

\end{document}
